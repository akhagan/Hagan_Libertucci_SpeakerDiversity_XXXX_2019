\documentclass[10pt,]{article}
\usepackage{lmodern}
\usepackage{amssymb,amsmath}
\usepackage{ifxetex,ifluatex}
\usepackage{fixltx2e} % provides \textsubscript
\ifnum 0\ifxetex 1\fi\ifluatex 1\fi=0 % if pdftex
  \usepackage[T1]{fontenc}
  \usepackage[utf8]{inputenc}
\else % if luatex or xelatex
  \ifxetex
    \usepackage{mathspec}
  \else
    \usepackage{fontspec}
  \fi
  \defaultfontfeatures{Ligatures=TeX,Scale=MatchLowercase}
\fi
% use upquote if available, for straight quotes in verbatim environments
\IfFileExists{upquote.sty}{\usepackage{upquote}}{}
% use microtype if available
\IfFileExists{microtype.sty}{%
\usepackage{microtype}
\UseMicrotypeSet[protrusion]{basicmath} % disable protrusion for tt fonts
}{}
\usepackage[margin=1.0in]{geometry}
\usepackage{hyperref}
\hypersetup{unicode=true,
            pdftitle={Policy should change to improve invited speaker diversity and reflect trainee diversity},
            pdfborder={0 0 0},
            breaklinks=true}
\urlstyle{same}  % don't use monospace font for urls
\usepackage{graphicx,grffile}
\makeatletter
\def\maxwidth{\ifdim\Gin@nat@width>\linewidth\linewidth\else\Gin@nat@width\fi}
\def\maxheight{\ifdim\Gin@nat@height>\textheight\textheight\else\Gin@nat@height\fi}
\makeatother
% Scale images if necessary, so that they will not overflow the page
% margins by default, and it is still possible to overwrite the defaults
% using explicit options in \includegraphics[width, height, ...]{}
\setkeys{Gin}{width=\maxwidth,height=\maxheight,keepaspectratio}
\IfFileExists{parskip.sty}{%
\usepackage{parskip}
}{% else
\setlength{\parindent}{0pt}
\setlength{\parskip}{6pt plus 2pt minus 1pt}
}
\setlength{\emergencystretch}{3em}  % prevent overfull lines
\providecommand{\tightlist}{%
  \setlength{\itemsep}{0pt}\setlength{\parskip}{0pt}}
\setcounter{secnumdepth}{0}
% Redefines (sub)paragraphs to behave more like sections
\ifx\paragraph\undefined\else
\let\oldparagraph\paragraph
\renewcommand{\paragraph}[1]{\oldparagraph{#1}\mbox{}}
\fi
\ifx\subparagraph\undefined\else
\let\oldsubparagraph\subparagraph
\renewcommand{\subparagraph}[1]{\oldsubparagraph{#1}\mbox{}}
\fi

%%% Use protect on footnotes to avoid problems with footnotes in titles
\let\rmarkdownfootnote\footnote%
\def\footnote{\protect\rmarkdownfootnote}

%%% Change title format to be more compact
\usepackage{titling}

% Create subtitle command for use in maketitle
\newcommand{\subtitle}[1]{
  \posttitle{
    \begin{center}\large#1\end{center}
    }
}

\setlength{\droptitle}{-2em}

  \title{\textbf{Policy should change to improve invited speaker diversity and
reflect trainee diversity}}
    \pretitle{\vspace{\droptitle}\centering\huge}
  \posttitle{\par}
    \author{}
    \preauthor{}\postauthor{}
    \date{}
    \predate{}\postdate{}
  
\usepackage{booktabs}
\usepackage{longtable}
\usepackage{array}
\usepackage{multirow}
\usepackage[table]{xcolor}
\usepackage{wrapfig}
\usepackage{float}
\usepackage{colortbl}
\usepackage{pdflscape}
\usepackage{tabu}
\usepackage{threeparttable}
\usepackage{threeparttablex}
\usepackage[normalem]{ulem}
\usepackage{makecell}
\usepackage{caption}

\usepackage{helvet} % Helvetica font
\renewcommand*\familydefault{\sfdefault} % Use the sans serif version of the font
\usepackage[T1]{fontenc}

\usepackage[none]{hyphenat}

\usepackage{setspace}
\doublespacing
\setlength{\parskip}{1em}

\usepackage{lineno}

\usepackage{pdfpages}
\floatplacement{figure}{H} % Keep the figure up top of the page

\begin{document}
\maketitle

\vspace{30mm}

Running title: Suggestions to improve invited speaker diversity

\vspace{35mm}

Ada K. Hagan, Ph.D.\({^1\dagger}\), Rebecca M. Pollet, Ph.D.\({^1}\),
and Josie Libertucci, Ph.D.\({^2}\)

\vspace{35mm}

\(\dagger\) To whom correspondence should be addressed:
\href{mailto:akhagan@umich.edu}{\nolinkurl{akhagan@umich.edu}}

1. Department of Microbiology \& Immunology, University of Michigan, Ann
Arbor, Michigan

2. Farncombe Family Digestive Health Research Institute, Department of
Medicine, McMaster University, Hamilton, Ontario, Canada

Figures: 1

Tables: 1

Financial support: Department of Microbiology \& Immunology, University
of Michigan

\newpage

\subsection{Conflicts of Interest}\label{conflicts-of-interest}

\vspace{40mm}

All authors affirm that there are no conflicts of interest.

\newpage

\linenumbers

\subsection{Abstract}\label{abstract}

The biomedical sciences have a problem retaining white women and
underrepresented minorities in academia. Despite increases in the
representation of these groups in faculty candidate pools, they are
still underrepresented at the faculty level, particularly at the Full
Professor level. The lack of diverse individuals at the Full Professor
level contributes to the attrition of women and under-represented
minorities, as it confirms unconscious biases. The presence of
unconscious biases contribute to feelings of not belonging by trainees
and are amplified by visual representation of who is presented as the
``top scientist in their field''. Top scientists are not only defined by
the attainment of Full Professorships, but also through invited seminar
series. Invitations for faculty to present their research at other
university departments is highly valued offer that provides an
opportunity for collaborations and networking. However, if invited
speakers do not represent the demographics of current trainees, these
visual representations of successful scientists may contribute to
decreased attitudes of self-identification as a scientist, ultimately
resulting in trainees leaving the field or the academy. In this study,
we compare invited-speaker demographics to the current trainee
demographics in one microbiology and immunology department and find that
trainees are not proportionally represented by speakers invited to the
department. Our investigation prompted changes in policy for how invited
speakers are selected in the future to invite a more diverse group of
scientists. To facilitate this process, we developed a set of tips and a
web-based resource that allows scientists, committees, and moderators to
identify members of under-served groups. These resources can be easily
adapted by other fields or sub-fields to promote inclusion and diversity
at seminar series', conferences, and colloquia.

\subsection{Keywords}\label{keywords}

inclusion, diversity, invited speakers, academia, graduate programs

\newpage

\subsection{Background}\label{background}

Long-standing systemic bias, sexism, and racism have contributed to the
under-representation of many racial and ethnic groups, as well as women,
in science, technology, engineering, and math (STEM) fields (1--4).
Specifically, within the field of biomedical research in the United
States, the proportion of underrepresented minorities at the full
professor level has remained consistently low at 4\% (survey data taken
from the NIH from 2001 to 2013), compared to the U.S. population, which
is 32.3\% (5, 6). Similar discrepancies exist for women in biomedical
sciences as full professorships are currently held mostly by men (7, 8).
As demographics of faculty within the biomedical sciences remains skewed
towards Caucasian men, the demographics of trainees (graduate students
and postdocs) are becoming more diverse (5).

Policy changes are needed to support inclusion of all individuals,
particularly in the biomedical sciences(9). To increase retention of
historically under-represented minorities (HURM), non-Caucasian/non-HURM
(NCNH) individuals, and white women in biomedical fields, it is
important for trainees to have visual representations of themselves as
scientists. The importance of representation in retaining a diverse
group of individuals in STEM fields is supported by social role theory
(10). Individuals make inferences about characteristics that are needed
to be successful in a given role by examining individuals that most
occupy that role (10, 11). However, there is a lack of diverse
scientific experts in academia so underrepresented minorities are not
seeing adequate visual representations of themselves at the faculty
level. Therefore, trainees who do not see representation of themselves
in senior faculty positions, may decide that they do not possess the
characteristics that are required to succeed.

Invited seminar series are common within biomedical departments across
the United States (12). Usually, seminar series' consist of faculty
members selecting a scientist from another institution to visit their
university and present their research, as well as meet with other
faculty members and trainees. Named lectureships follow the same format
but are decided by committee and are considered more prestigious because
they are named in honor of prominent local scientists. These seminar
series and lectureships provide an opportunity for trainees to be
exposed to research outside of their department. Additionally, being an
invited speaker provides the scientist with an opportunity to make
future collaborations and build their own \emph{curriculum vitae} (CV).
Scientists invited to give seminars are widely regarded as successful
and the top in their field. Thus, if trainees are constantly being
exposed to ``the top scientist in their field'', according to social
role theory, it signals who is successful in that field. While some have
examined this issue by studying and promoting the inclusion of more
women speakers at conferences, how department speaker series compare to
the trainee diversity of that department is unknown (13--15).

In this study, we examine and compare the proportion of HURM, NCNH, and
women invited speakers to white men in the Department of Microbiology
and Immunology at the University of Michigan. Additionally, we compare
invited-speaker demographics to the current trainee demographics as a
means to gauge if trainee demographics are being represented accordingly
throughout the seminar series. Following our investigation, we proposed
a policy change to the Department of Microbiology and Immunology in how
invited speakers are selected as a means to promote inclusion in our
department and reduce unconscious bias. In order to facilitate inviting
a more diverse group of scientists, we developed a set of resources that
allow scientists, within the fields of microbiology and immunology, to
self-identify as having an under-represented or under-served identity
including: HURM, non-Caucasian/non-HURM, or a white woman. These
resources will promote inclusion and diversity by providing greater
representation of all scientists and will provide hosts an opportunity
to invite a more diverse group of scientists.

\subsection{Methods}\label{methods}

Each academic year, each faculty member in the Department of
Microbiology and Immunology at the University of Michigan has the
opportunity to invite one speaker per year for a weekly seminar series.
Some of these seminar slots are dedicated to named lectureships, which
are decided by committee, and three trainee-invited speakers. We
analyzed the demographics of invited speakers and faculty hosts for five
academic years (Fall 2014 - Spring 2019), and compared them to the
current trainees when the data were analyzed (Spring 2019). Each speaker
was only counted once and those listed as departmental faculty members
or as a ``host'' at any point could not also be considered ``invited
speakers''. The list of faculty hosts was used as a proxy for faculty
demographics since as hosts, these faculty members are visible
representatives of the department. The trainees were identified using
departmental email lists that included masters students, doctoral
students, and post-doctoral fellows.

This is a retrospective study, thus speakers were not asked for their
identities at the time of visit. Instead we hand-coded proxy
demographics using personal knowledge, photos, and CVs. The presenting
gender of each individual was assigned using a binary system
(man/woman). Due to the low number of individuals in the study,
race/ethnicity demographics were split in three groups: Caucasian,
Historically Under-represented Minority (HURM), and
Non-Caucasian/Non-HURM (NCNH), each with a binary (yes/no) possibility.
Caucasian was assigned using the current U.S. Census definition where
those of Middle Eastern, European, and Russian descent are included.
HURM individuals were restricted to those with African-American,
Indigenous, Alaskan/Hawaiian Native, Latinx and/or Hispanic heritage.
All others were placed into the NCNH group. We recognize that our proxy
demographics are a limitation of the analysis and want to acknowledge
that biological sex (male/female) is not always equivalent to the gender
that an individual presents as (man/woman), which is also distinct from
the gender(s) that an individual self-identifies as. We also want to
acknowledge that there are many other identities that are not captured
in this limited analysis.

Data were analyzed and figures generated in R Statistical Software,
using relevant packages (16--28).

\subsection{Results}\label{results}

To understand the representation of women, we compared the proportion of
women in each academic role. At the trainee level, more than half of
students and postdoctoral fellows were women. That dropped to 46.77\% of
faculty hosts and 38.73\% of the invited speakers (Fig. 1A). Of 27
lectureships over the five year period, 37.04\% were awarded to women.

Our analysis identified an over-representation of Caucasian individuals
as hosting faculty and invited speakers (80\% each), relative to the
proportion of Caucasian trainees, which was 55\% (Fig. 1B). We also
observed declines in the representation of HURM and NCNH faculty and
speakers relative to the trainees (Fig 1B). HURM trainees made up 11\%
of the department, on track with the 11\% of U.S. microbiology and
immunology doctorates awarded in 2017 (29). However, only 8.5\% of
invited speakers, and none of the hosting faculty, were HURM scientists.
NCNH trainees were 34\% of department students and postdocs (versus 22\%
of U.S. microbiology and immunology doctorates in 2017), but only 19\%
of hosting faculty and 10.5\% of invited speakers (29).

The more prestigious invited speaker lectureships were also dominated by
Caucasian scientists, who comprised 81.48\% of those awarded (Fig. 1C).
HURM and NCNH scientists were awarded 3 and 2 lectureships,
respectively. Because the intersection of identities can compound biases
and outcomes, we further examined the lectureships by gender and
race/ethnicity status (30). Caucasian men and women accounted for
44.44\% and 37.04\% of the lectureships, respectively. Just 18.52\% of
lectureships were held by non-Caucasian men while none were held by
non-Caucasian women (Fig. 1D).

\subsection{Discussion}\label{discussion}

This study found that the proportion of HURM and NCNH invited speakers
were under-representative of the trainee populations for each group.
Additionally, within the last 5 years, no HURM or NCNH woman was awarded
a lectureship. This means that the department is not providing
non-Caucasian trainees with adequate representation of successful
scientists. Taking this into context of social role theory, by not
adequately representing the diversity of all trainees, the department is
not supporting an inclusive environment in terms of visual faculty
representation. We also found that the proportion of women as faculty
hosts and speakers in our study population is equivalent to global
estimates that 40\% of microbiologists are women, though women only
represent about 30\% of academic biomedical faculty (7, 31). Women are
also over-represented as graduate students and postdoctoral fellows in
this department. Overall, Caucasian scientists are over-represented as
host faculty and invited speakers, compared to their presence as
trainees, particularly when lectureships were considered.

Several papers have investigated the representation of women at
scientific conferences, however, we have only identified one that
focused on invited speakers at universities (12). In their study,
Nittrouer et al, examined 3,652 talks at 50 U.S. institutions in 2013 -
2014 and found that women faculty are less likely to be invited
speakers, despite similar acceptance rates (12). We have not been able
to identify any publications examining scientific speaker diversity
beyond gender. This seems to be the first, which is concerning since
conclusions drawn from gender-based studies are often framed, and
considered, to be applicable to other marginalized groups (e.g., HURM).
This is a flawed assumption. While there is no doubt some overlap, each
group remains marginalized due to a unique complex set of factors that
cannot always be solved by gender-based solutions. U.S. institutions,
such as the University of Michigan have a particular responsibility to
the historically suppressed populations included in our definition of
HURMs. We therefore implore U.S. institutions to apply this framing to
their discussions and research.

Departments have different processes and criteria for selecting invited
speakers, but it is a matter of pride to bring the best scientists
possible. It may be that the definition of ``best'' poses a problem to
under-represented and under-served groups (e.g., white women, HURM, and
Asian) who are held to stricter competency standards and report having
to work harder than white men to be perceived as legitimate scholars
(32, 33). Some departments only invite tenured faculty, which severely
limits the number of potential speakers who are white women or
non-Caucasian. Yet, another scenario is that pre-tenure faculty members
invite prestigious, tenured faculty in their field to network and secure
letters for their own tenure package. The increased burden of white
women and non-Caucasian scientists to prove competency decreases their
likelihood to be considered for either tenure or as possible source of
tenure letters.

Each underrepresented group in our cohort faces a complex set of
barriers to achieving faculty status. For instance, the decision to
invite a woman may also be negatively impacted by assumptions about
competency and dedication. The dedication of women who have children to
their work is perceived to be less than that of their colleagues, i.e.,
men who also have children (34--36). The perceived prioritization and
commitments of women to family over work may cause faculty to doubt
their acceptance of a speaking invitation, despite the prestigious
nature of these invitations and evidence that men and women accept at
similar rates (12, 37). As a result, the faculty member may invite a
different colleague who they feel is more likely to agree (and is a
man). Another large portion of our sample were the NCNH cohort, who are
predominately Asian/Asian American individuals. Although Asian
scientists are well-represented in the US scientific workforce, they
face significant bias and barriers to inclusion in society and academia
(38, 39). For instance, despite the higher employment rate of Asian
scientists, they were not well-represented in the more prestigious
lectureships.

While HURM and NCNH share some experiences, differences including
varying rates of hiring and tenure promotion mean unique considerations
are important for inclusion of each group (3). For instance, a major
barrier to inclusion of HURM faculty at similar proportions to HURM
trainees is the low transition rate of scientists from HURM backgrounds
to faculty positions and the associated low proportion of HURM faculty
(40). The proportion of HURM faculty at the Assistant and Associate
Professor level is currently higher than at Full Professor so it will be
difficult to increase speaker diversity if early-career researchers are
not being considered (41). Increased performance expectations and
patterns of exclusions are consistent themes in studies characterizing
the HURM faculty experience (42, 43). Therefore, inclusion of HURM
faculty in seminar series is likely essential to increasing the number
of HURM Associate and Full Professors. Even when HURM speaker rates
match the proportion of HURM faculty employment, HURM trainees will be
represented at a significantly higher proportion. Inclusion of HURM
faculty in these seminar series is just one aspect of larger
institutional change that is needed (44).

\subsection{Instituting Policy Change}\label{instituting-policy-change}

In an attempt to promote inclusion within the Department of Microbiology
and Immunology at the University of Michigan, these data were presented
to faculty members and the department chair (Dr.~Mobley). Since trainee
demographics were not represented by the seminar speaker demographics
over the past 5 years, we proposed a policy change as to how seminar
speakers were being invited. One suggestion was to switch from
faculty-invited to lab-invited speakers in an attempt to allow trainees
to choose a speaker that best represented themselves (Table 1).

The implicit biases that affect perceptions of marginalized groups are
an issue, but we must acknowledge that it is not always possible to
identify members of historically under-served communities. For instance,
after data analysis, we learned that at least one speaker in our data
set should have been categorized as a HURM instead of Caucasian, but it
wasn't readily apparent from their internet presence or CV. This
limitation makes two important points: that perceived identity often
plays a larger role than self-identification, and that we need better
tools to identify members of marginalized groups. Another policy
suggestion is for departments to invite their speakers to spend time
discussing their personal journeys through science, in addition to their
scientific stories (Table 1). This would enable those who wish, to
discuss how their identity(ies) interacted with their careers. In
addition to these suggestions for policy change, we have created
resources that allow scientists to self-identify as under-served groups
and thus provide host faculty with more diverse choices (Table 1).

\subsection{Building Diversify}\label{building-diversify}

Motivated by a lack of resources to identify scientists who are members
of marginalized and/or historically under-served groups, and inspired by
resources in other fields--DiversifyEEB and DiversifyChemistry--we
created DiversifyMicrobiology and DiversifyImmunology (45--48). These
resources are a tool for symposium organizers, award committees, search
committees, and other scientists to identify individuals to diversify
their pools. Additionally, we have built these as a template to be used
by other fields and organizations that wish to create their own lists.
Since these lists are compiled by self-nomination, we can ensure that
only scientists comfortable revealing their marginalized identities are
included.

The self-nomination form is a Google Form with entries logged in a
private Google Sheet. This form is embedded within the website and can
be linked to directly. The use of a Google Forms allows us to maintain
this database at no cost and gives us the flexibility to add questions
or change response options without disrupting previous responses.
Entries are logged in a private spreadsheet so that entries can be
screened before being added to the public database. This screening
includes two steps: confirming that each person is listed in the
database only once and that any submitted website is a personal,
professional website. If both criteria are met, a new entry is added to
the public database spreadsheet. If a person is already listed in the
database, their information is updated to the most recent submission.

This public spreadsheet is embedded in the website and can be open
separately as a locked (uneditable) Google Sheet, allowing the list to
be easily searched. We have chosen to list individuals' academic
information first in the spreadsheet to encourage a focus on academic
achievement rather than tokenization of marginalized identities.
Currently the database lists individuals in order of self-nomination but
future versions will be re-sorted based on name and/or academic field to
varying the individuals who may receive more attention for simply being
at the top of the list.

The website provides an interface to the Google forms and spreadsheets
with template pages for viewing the list, adding a name to the list, and
finding additional resources. Importantly, our website creation tool is
hosted for free by GitHub, which provides a free website for each GitHub
organization. Basic tools and skills required to set up a Diversify site
include knowledge of, or experience with, the version control tool git,
the web-tool GitHub, and a text editor. A tutorial in the
DiversifyMicrobiology repository on GitHub provides links to these
resources and instructions for adapting the tool to your own field (47).

\subsection{Conclusion}\label{conclusion}

To increase the retention of white women, HURM and NCNH trainees in the
biomedical sciences, they must also be represented as experts. However,
the invited speaker diversity at one department does not represent the
diversity of trainees. To facilitate the identification and recruitment
of individuals in these historically under-served groups, we have built
a tool to create self-nominated, field-specific lists.

\subsection{Acknowledgements}\label{acknowledgements}

We thank Drs. Beth Moore and Harry Mobley and the Department of
Microbiology \& Immunology, University of Michigan for their input and
financial support that enabled publication of our manuscript. We thank
Bonnie Krey and former speaker series coordinators Drs. Nicole
Koropatkin and Kathy Spindler for access to invited speaker data. We
would also like to acknowledge and thank Nick Lesniak and Dr.~Ariangela
Kozick for their comments and suggestions.

\subsection{Author Contributions}\label{author-contributions}

A.K.H. collected the data, assigned demographics, analyzed the data,
created the website, and wrote the methods and results. R.M.P. created
the Google lists, forms, and website content and the description of
their maintenance. J.L. wrote the introduction and provided conceptual
advice. A.K.H. and J.L facilitated the policy change to the UM
Department of Microbiology and Immunology. All authors contributed to
preparing the final manuscript.

\subsection{Code and data
availability}\label{code-and-data-availability}

The anonymized data, code for all analysis steps, and an Rmarkdown
version of this manuscript is available at
\url{https://github.com/akhagan/Hagan_SpeakerDiversity_JMBE_2019}.
Template and complete instructions for generating a field-specific
Diversity website are available at
\url{https://github.com/diversifymicrobiology/DiversifyMicrobiology.github.io/}.

\newpage

\begin{center}
\captionof{table}{List of suggestions and resources to increase invited speaker diversity.}
\small
\begin{tabular}{|l|l|l|}
\hline

\rowcolor{lightgray}
\textbf{Suggestion} & \textbf{Description} & \textbf{Resource} \\ \hline

Lab-invited speakers & \makecell[l]{Faculty members can request \\suggestions from trainees} & \\ \hline

Use a list & \makecell[l]{Many lists of scientists from \\under-represented and under-served \\groups are available} &  \makecell[l]{https://DiversifyMicrobiology.\\github.io/resources}\\ \hline

Create a list & \makecell[l]{Use the GitHub template \\ create a self-nomination list and \\resource for your field} & \makecell[l]{https://github.com/diversifymicrobiology/\\DiversifyMicrobiology.github.io} \\ \hline

Highlight the journey & \makecell[l]{Invite all speakers to spend \\a few moments describing their \\personal science journey} & \\ \hline

\end{tabular}
\end{center}

\begin{figure}
\centering
\includegraphics{Figure_1.png}
\caption{\textbf{The demographics of invited speakers, hosting faculty,
and trainees.} A) The proportion of women in each academic role. B) The
proportion of each academic role represented by individuals that are
Caucasian (left), Historically Underrepresented Minorities (HURM,
center) or Non-Caucasian/Non-HURM (NCNH, right). C-D)The percent of
lectureships awarded to individuals that are C) Caucasian, HURM, or NCNH
and D) Caucasian, HURM, or NCNH by gender.}
\end{figure}

\newpage

\subsection{References}\label{references}

\hypertarget{refs}{}
\hypertarget{ref-Martinez2018}{}
1. \textbf{Martinez LR}, \textbf{Boucaud DW}, \textbf{Casadevall A},
\textbf{August A}. 2018. Factors contributing to the success of
NIH-designated underrepresented minorities in academic and nonacademic
research positions. CBELife Sciences Education \textbf{17}:ar32.
doi:\href{https://doi.org/10.1187/cbe.16-09-0287}{10.1187/cbe.16-09-0287}.

\hypertarget{ref-AllenRamdial2014}{}
2. \textbf{Allen-Ramdial S-AA}, \textbf{Campbell AG}. 2014. Reimagining
the pipeline: Advancing STEM diversity, persistence, and success.
BioScience \textbf{64}:612--618.
doi:\href{https://doi.org/10.1093/biosci/biu076}{10.1093/biosci/biu076}.

\hypertarget{ref-Fang2000}{}
3. \textbf{Fang D}. 2000. Racial and ethnic disparities in faculty
promotion in academic medicine. JAMA \textbf{284}:1085.
doi:\href{https://doi.org/10.1001/jama.284.9.1085}{10.1001/jama.284.9.1085}.

\hypertarget{ref-Gibbs2014}{}
4. \textbf{Gibbs KD}, \textbf{McGready J}, \textbf{Bennett JC},
\textbf{Griffin K}. 2014. Biomedical science ph.D. career interest
patterns by race/ethnicity and gender. PLoS ONE \textbf{9}:e114736.
doi:\href{https://doi.org/10.1371/journal.pone.0114736}{10.1371/journal.pone.0114736}.

\hypertarget{ref-Meyers2018}{}
5. \textbf{Meyers LC}, \textbf{Brown AM}, \textbf{Moneta-Koehler L},
\textbf{Chalkley R}. 2018. Survey of checkpoints along the pathway to
diverse biomedical research faculty. PLOS ONE \textbf{13}:e0190606.
doi:\href{https://doi.org/10.1371/journal.pone.0190606}{10.1371/journal.pone.0190606}.

\hypertarget{ref-nsf_2014}{}
6. \textbf{National Center for Science and Engineering Statistics}.
2014. Women, minorities, and persons with disabilities in science and
engineering. National Science Foundation, Alexandria, VA.

\hypertarget{ref-Jena2015}{}
7. \textbf{Jena AB}, \textbf{Khullar D}, \textbf{Ho O}, \textbf{Olenski
AR}, \textbf{Blumenthal DM}. 2015. Sex differences in academic rank in
US medical schools in 2014. JAMA \textbf{314}:1149.
doi:\href{https://doi.org/10.1001/jama.2015.10680}{10.1001/jama.2015.10680}.

\hypertarget{ref-Rotbart2012}{}
8. \textbf{Rotbart HA}, \textbf{McMillen D}, \textbf{Taussig H},
\textbf{Daniels SR}. 2012. Assessing gender equity in a large academic
department of pediatrics. Academic Medicine \textbf{87}:98--104.
doi:\href{https://doi.org/10.1097/acm.0b013e31823be028}{10.1097/acm.0b013e31823be028}.

\hypertarget{ref-Coe2019}{}
9. \textbf{Coe IR}, \textbf{Wiley R}, \textbf{Bekker L-G}. 2019.
Organisational best practices towards gender equality in science and
medicine. The Lancet \textbf{393}:587--593.
doi:\href{https://doi.org/10.1016/s0140-6736(18)33188-x}{10.1016/s0140-6736(18)33188-x}.

\hypertarget{ref-Eagly1984}{}
10. \textbf{Eagly AH}, \textbf{Steffen VJ}. 1984. Gender stereotypes
stem from the distribution of women and men into social roles. Journal
of Personality and Social Psychology \textbf{46}:735--754.
doi:\href{https://doi.org/10.1037/0022-3514.46.4.735}{10.1037/0022-3514.46.4.735}.

\hypertarget{ref-Carter2018}{}
11. \textbf{Carter AJ}, \textbf{Croft A}, \textbf{Lukas D},
\textbf{Sandstrom GM}. 2018. Women's visibility in academic seminars:
Women ask fewer questions than men. PLOS ONE \textbf{13}:e0202743.
doi:\href{https://doi.org/10.1371/journal.pone.0202743}{10.1371/journal.pone.0202743}.

\hypertarget{ref-nittrouer_gender_2018}{}
12. \textbf{Nittrouer CL}, \textbf{Hebl MR}, \textbf{Ashburn-Nardo L},
\textbf{Trump-Steele RCE}, \textbf{Lane DM}, \textbf{Valian V}. 2018.
Gender disparities in colloquium speakers at top universities.
Proceedings of the National Academy of Sciences \textbf{115}:104--108.
doi:\href{https://doi.org/10.1073/pnas.1708414115}{10.1073/pnas.1708414115}.

\hypertarget{ref-kalejta_gender_2017}{}
13. \textbf{Kalejta RF}, \textbf{Palmenberg AC}. 2017. Gender Parity
Trends for Invited Speakers at Four Prominent Virology Conference
Series. Journal of Virology \textbf{91}.
doi:\href{https://doi.org/10.1128/JVI.00739-17}{10.1128/JVI.00739-17}.

\hypertarget{ref-casadevall_presence_2014}{}
14. \textbf{Casadevall A}, \textbf{Handelsman J}. 2014. The Presence of
Female Conveners Correlates with a Higher Proportion of Female Speakers
at Scientific Symposia. mBio \textbf{5}.
doi:\href{https://doi.org/10.1128/mBio.00846-13}{10.1128/mBio.00846-13}.

\hypertarget{ref-klein_speaking_2017}{}
15. \textbf{Klein RS}, \textbf{Voskuhl R}, \textbf{Segal BM},
\textbf{Dittel BN}, \textbf{Lane TE}, \textbf{Bethea JR}, \textbf{Carson
MJ}, \textbf{Colton C}, \textbf{Rosi S}, \textbf{Anderson A},
\textbf{Piccio L}, \textbf{Goverman JM}, \textbf{Benveniste EN},
\textbf{Brown MA}, \textbf{Tiwari-Woodruff SK}, \textbf{Harris TH},
\textbf{Cross AH}. 2017. Speaking out about gender imbalance in invited
speakers improves diversity. Nature Immunology \textbf{18}:475--478.
doi:\href{https://doi.org/10.1038/ni.3707}{10.1038/ni.3707}.

\hypertarget{ref-R_software_2017}{}
16. \textbf{R Core Team}. 2017. R: A language and environment for
statistical computing. R Foundation for Statistical Computing, Vienna,
Austria.

\hypertarget{ref-wickham_tidyverse_2017}{}
17. \textbf{Wickham H}. 2017. Tidyverse: Easily Install and Load the
'Tidyverse'.

\hypertarget{ref-cowplot}{}
18. \textbf{Wilke CO}. 2019. Cowplot: Streamlined plot theme and plot
annotations for 'ggplot2'.

\hypertarget{ref-markdown}{}
19. \textbf{Allaire J}, \textbf{Horner J}, \textbf{Xie Y}, \textbf{Marti
V}, \textbf{Porte N}. 2018. Markdown: 'Markdown' rendering for r.

\hypertarget{ref-rmd_book}{}
20. \textbf{Xie Y}, \textbf{Allaire J}, \textbf{Grolemund G}. 2018. R
markdown: The definitive guide. Chapman; Hall/CRC, Boca Raton, Florida.

\hypertarget{ref-rmd_rstudio}{}
21. \textbf{Allaire J}, \textbf{Xie Y}, \textbf{McPherson J},
\textbf{Luraschi J}, \textbf{Ushey K}, \textbf{Atkins A},
\textbf{Wickham H}, \textbf{Cheng J}, \textbf{Chang W}, \textbf{Iannone
R}. 2018. Rmarkdown: Dynamic documents for r.

\hypertarget{ref-knitr_2014}{}
22. \textbf{Xie Y}. 2014. Knitr: A comprehensive tool for reproducible
research in R. \emph{In} Stodden, V, Leisch, F, Peng, RD (eds.),
Implementing reproducible computational research. Chapman; Hall/CRC.

\hypertarget{ref-knitr_2018}{}
23. \textbf{Xie Y}. 2018. Knitr: A general-purpose package for dynamic
report generation in r.

\hypertarget{ref-lubridate}{}
24. \textbf{Grolemund G}, \textbf{Wickham H}. 2011. Dates and times made
easy with lubridate. Journal of Statistical Software \textbf{40}:1--25.

\hypertarget{ref-readxl}{}
25. \textbf{Wickham H}, \textbf{Bryan J}. 2018. Readxl: Read excel
files.

\hypertarget{ref-pdftools}{}
26. \textbf{Ooms J}. 2019. Pdftools: Text extraction, rendering and
converting of pdf documents.

\hypertarget{ref-wickham_scales_2018}{}
27. \textbf{Wickham H}. 2018. Scales: Scale Functions for Visualization.

\hypertarget{ref-neuwirth_rcolorbrewer_2014}{}
28. \textbf{Neuwirth E}. 2014. RColorBrewer: ColorBrewer Palettes.

\hypertarget{ref-nsf_survey_2017}{}
29. \textbf{National Center for Science and Engineering Statistics}.
2017. Survey of Doctorate Recipients, Survey Year 2017. National Science
Foundation, Alexandria, VA.

\hypertarget{ref-Crenshaw1989}{}
30. \textbf{Crenshaw K}. 1989. Demarginalizing the Intersection of Race
and Sex: A black feminist critique of antidiscrimination doctrine,
feminist theory and antiracist politics. University of Chicago Legal
Forum \textbf{1989}.
doi:\href{https://doi.org/10.1007/s11162-008-9097-4}{10.1007/s11162-008-9097-4}.

\hypertarget{ref-allagnat_gender_2017}{}
31. \textbf{Allagnat L}, \textbf{Berghmans S}, \textbf{Falk-Krzesinski
HJ}, \textbf{Hanafi S}, \textbf{Herbert R}, \textbf{Huggett S},
\textbf{Tobin S}. 2017. Gender in the global research landscape.

\hypertarget{ref-BlairLoy2017}{}
32. \textbf{Blair-Loy M}, \textbf{Rogers L}, \textbf{Glaser D},
\textbf{Wong Y}, \textbf{Abraham D}, \textbf{Cosman P}. 2017. Gender in
engineering departments: Are there gender differences in interruptions
of academic job talks? Social Sciences \textbf{6}:29.
doi:\href{https://doi.org/10.3390/socsci6010029}{10.3390/socsci6010029}.

\hypertarget{ref-noauthor_seeking_2013}{}
33. \textbf{National Research Council \textnormal{Policy and Global
Affairs}}, \textbf{Committee on Women in Science, Engineering, and
Medicine}, \textbf{Committee on Advancing Institutional Transformation
for Minority Women in Academia}, \textbf{Rapporteur KM}. 2013. Seeking
Solutions: Maximizing American Talent by Advancing Women of Color in
Academia: Summary of a Conference. National Academies Press, Washington,
D.C.

\hypertarget{ref-Firth1982}{}
34. \textbf{Firth M}. 1982. Sex discrimination in job opportunities for
women. Sex Roles \textbf{8}:891--901.
doi:\href{https://doi.org/10.1007/bf00287858}{10.1007/bf00287858}.

\hypertarget{ref-Correll2007}{}
35. \textbf{Correll SJ}, \textbf{Benard S}, \textbf{Paik I}. 2007.
Getting a job: Is there a motherhood penalty? American Journal of
Sociology \textbf{112}:1297--1339.
doi:\href{https://doi.org/10.1086/511799}{10.1086/511799}.

\hypertarget{ref-Fuegen2004}{}
36. \textbf{Fuegen K}, \textbf{Biernat M}, \textbf{Haines E},
\textbf{Deaux K}. 2004. Mothers and fathers in the workplace: How gender
and parental status influence judgments of job-related competence.
Journal of Social Issues \textbf{60}:737--754.
doi:\href{https://doi.org/10.1111/j.0022-4537.2004.00383.x}{10.1111/j.0022-4537.2004.00383.x}.

\hypertarget{ref-xu_gender_2008}{}
37. \textbf{Xu YJ}. 2008. Gender Disparity in STEM Disciplines: A Study
of Faculty Attrition and Turnover Intentions. Research in Higher
Education \textbf{49}:607--624.
doi:\href{https://doi.org/10.1007/s11162-008-9097-4}{10.1007/s11162-008-9097-4}.

\hypertarget{ref-Hwang2008}{}
38. \textbf{Hwang W-C}, \textbf{Goto S}. 2008. The impact of perceived
racial discrimination on the mental health of asian american and latino
college students. Cultural Diversity and Ethnic Minority Psychology
\textbf{14}:326--335.
doi:\href{https://doi.org/10.1037/1099-9809.14.4.326}{10.1037/1099-9809.14.4.326}.

\hypertarget{ref-Tran2019}{}
39. \textbf{Tran VC}, \textbf{Lee J}, \textbf{Huang TJ}. 2019.
Revisiting the asian second-generation advantage. Ethnic and Racial
Studies \textbf{42}:2248--2269.
doi:\href{https://doi.org/10.1080/01419870.2019.1579920}{10.1080/01419870.2019.1579920}.

\hypertarget{ref-Gibbs2016}{}
40. \textbf{Gibbs KD}, \textbf{Basson J}, \textbf{Xierali IM},
\textbf{Broniatowski DA}. 2016. Decoupling of the minority PhD talent
pool and assistant professor hiring in medical school basic science
departments in the US. eLife \textbf{5}.
doi:\href{https://doi.org/10.7554/elife.21393}{10.7554/elife.21393}.

\hypertarget{ref-whittaker_retention_2015}{}
41. \textbf{Whittaker JA}, \textbf{Montgomery BL}, \textbf{Martinez
Acosta VG}. 2015. Retention of Underrepresented Minority Faculty:
Strategic Initiatives for Institutional Value Proposition Based on
Perspectives from a Range of Academic Institutions. Journal of
undergraduate neuroscience education: JUNE: a publication of FUN,
Faculty for Undergraduate Neuroscience \textbf{13}:A136--145.

\hypertarget{ref-pololi_race_2010}{}
42. \textbf{Pololi L}, \textbf{Cooper LA}, \textbf{Carr P}. 2010. Race,
Disadvantage and Faculty Experiences in Academic Medicine. Journal of
General Internal Medicine \textbf{25}:1363--1369.
doi:\href{https://doi.org/10.1007/s11606-010-1478-7}{10.1007/s11606-010-1478-7}.

\hypertarget{ref-hassouneh_experiences_2014}{}
43. \textbf{Hassouneh D}, \textbf{Lutz KF}, \textbf{Beckett AK},
\textbf{Junkins EP}, \textbf{Horton LL}. 2014. The experiences of
underrepresented minority faculty in schools of medicine. Medical
Education Online \textbf{19}:24768.
doi:\href{https://doi.org/10.3402/meo.v19.24768}{10.3402/meo.v19.24768}.

\hypertarget{ref-johnson_msphere_2019}{}
44. \textbf{Johnson MDL}. 2019. mSphere of Influence: Hiring of
Underrepresented Minority Assistant Professors in Medical School Basic
Science Departments Has a Long Way To Go. mSphere \textbf{4}.
doi:\href{https://doi.org/10.1128/mSphere.00599-19}{10.1128/mSphere.00599-19}.

\hypertarget{ref-Baucom_2019}{}
45. \textbf{Baucom R}, \textbf{Duffy M}. 2019. DiversifyEEB.
\url{https://diversifyeeb.com}.

\hypertarget{ref-Duffy_2019}{}
46. \textbf{Duffy M}, \textbf{McNeil AJ}. 2019. DiversifyChemistry.
\url{https://diversifychemistry.com}.

\hypertarget{ref-Hagan2019_micro}{}
47. \textbf{Hagan AK}, \textbf{Pollet RM}. 2019. DiversifyMicrobiology.
GitHub repository.
\url{https://github.com/diversifymicrobiology.github.io}; GitHub.

\hypertarget{ref-Hagan2019_immuno}{}
48. \textbf{Hagan AK}, \textbf{Pollet RM}. 2019. DiversifyImmunology.
GitHub repository.
\url{https://github.com/diversifyimmunology.github.io}; GitHub.


\end{document}
