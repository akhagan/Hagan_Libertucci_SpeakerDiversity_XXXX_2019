\documentclass[]{article}
\usepackage{lmodern}
\usepackage{amssymb,amsmath}
\usepackage{ifxetex,ifluatex}
\usepackage{fixltx2e} % provides \textsubscript
\ifnum 0\ifxetex 1\fi\ifluatex 1\fi=0 % if pdftex
  \usepackage[T1]{fontenc}
  \usepackage[utf8]{inputenc}
\else % if luatex or xelatex
  \ifxetex
    \usepackage{mathspec}
  \else
    \usepackage{fontspec}
  \fi
  \defaultfontfeatures{Ligatures=TeX,Scale=MatchLowercase}
\fi
% use upquote if available, for straight quotes in verbatim environments
\IfFileExists{upquote.sty}{\usepackage{upquote}}{}
% use microtype if available
\IfFileExists{microtype.sty}{%
\usepackage{microtype}
\UseMicrotypeSet[protrusion]{basicmath} % disable protrusion for tt fonts
}{}
\usepackage[margin=1.0in]{geometry}
\usepackage{hyperref}
\hypersetup{unicode=true,
            pdfborder={0 0 0},
            breaklinks=true}
\urlstyle{same}  % don't use monospace font for urls
\usepackage{graphicx,grffile}
\makeatletter
\def\maxwidth{\ifdim\Gin@nat@width>\linewidth\linewidth\else\Gin@nat@width\fi}
\def\maxheight{\ifdim\Gin@nat@height>\textheight\textheight\else\Gin@nat@height\fi}
\makeatother
% Scale images if necessary, so that they will not overflow the page
% margins by default, and it is still possible to overwrite the defaults
% using explicit options in \includegraphics[width, height, ...]{}
\setkeys{Gin}{width=\maxwidth,height=\maxheight,keepaspectratio}
\IfFileExists{parskip.sty}{%
\usepackage{parskip}
}{% else
\setlength{\parindent}{0pt}
\setlength{\parskip}{6pt plus 2pt minus 1pt}
}
\setlength{\emergencystretch}{3em}  % prevent overfull lines
\providecommand{\tightlist}{%
  \setlength{\itemsep}{0pt}\setlength{\parskip}{0pt}}
\setcounter{secnumdepth}{0}
% Redefines (sub)paragraphs to behave more like sections
\ifx\paragraph\undefined\else
\let\oldparagraph\paragraph
\renewcommand{\paragraph}[1]{\oldparagraph{#1}\mbox{}}
\fi
\ifx\subparagraph\undefined\else
\let\oldsubparagraph\subparagraph
\renewcommand{\subparagraph}[1]{\oldsubparagraph{#1}\mbox{}}
\fi

%%% Use protect on footnotes to avoid problems with footnotes in titles
\let\rmarkdownfootnote\footnote%
\def\footnote{\protect\rmarkdownfootnote}

%%% Change title format to be more compact
\usepackage{titling}

% Create subtitle command for use in maketitle
\newcommand{\subtitle}[1]{
  \posttitle{
    \begin{center}\large#1\end{center}
    }
}

\setlength{\droptitle}{-2em}

  \title{}
    \pretitle{\vspace{\droptitle}}
  \posttitle{}
    \author{}
    \preauthor{}\postauthor{}
    \date{}
    \predate{}\postdate{}
  
\usepackage{booktabs}
\usepackage{longtable}
\usepackage{array}
\usepackage{multirow}
\usepackage[table]{xcolor}
\usepackage{wrapfig}
\usepackage{float}
\usepackage{colortbl}
\usepackage{pdflscape}
\usepackage{tabu}
\usepackage{threeparttable}
\usepackage{threeparttablex}
\usepackage[normalem]{ulem}
\usepackage{makecell}
\usepackage{caption}

\usepackage{helvet} % Helvetica font
\renewcommand*\familydefault{\sfdefault} % Use the sans serif version of the font
\usepackage[T1]{fontenc}

\usepackage[none]{hyphenat}

\usepackage{setspace}
\doublespacing
\setlength{\parskip}{1em}

\usepackage{lineno}

\usepackage{pdfpages}
\floatplacement{figure}{H} % Keep the figure up top of the page

\begin{document}

\begin{center}
\captionof{table}{List of suggestions and resources to increase invited speaker diversity.}
\small
\centering
\begin{tabular}{>{\centering\arraybackslash}m{4cm}|>{\centering\arraybackslash}m{7cm}|>{}m{5cm}}

\hline

\rowcolor{lightgray}
\textbf{Suggestion} & \textbf{Description} & \textbf{Resource} \\\hline

Trainee-invited speakers & Request suggestions from trainees, increase number of trainee-group-invited speakers & \\\hline

Use a list & Lists of scientists from under-represented and under-served groups are available in several fields &  https://DiversifyMicrobiology .github.io/resources \\\hline

Create a list & Use the GitHub template to create a self-nomination list and resource for your field & https://github.com /diversifymicrobiology /DiversifyMicrobiology.github.io \\\hline

Use resources from professional societies & Many scientific societies have a committee focused on serving individuals from under-represented and underserved backgrounds. Other societies (e.g., SACNAS) are dedicated to these issues. & SACNAS, ABRCMS, AISES, ASM Subcommittee on Minority Education \\\hline

Think outside your sub-discipline & Speakers may introduce you to a technique that is not used in your sub-discipline & \\\hline

Consider scientists outside research-focused universities & Scientists from industry, teaching-focused institutions, and non-profit orgs have different approaches to their research & \\\hline

Communicate invitation expectations & Unit leadership should explicitly communicate expectations about who is invited to speak and the desired atmosphere & \\\hline

Encourage trainees to engage & \makecell[l]{When a talk is over, ensure that \\trainees are the first to ask questions} & \\\hline

Foster an inclusive atmosphere & Consider the identities of individuals the speaker is meeting with. Ask if the speaker would like to meet a particular student group & \\\hline

Highlight the journey & Invite speakers to spend a few moments describing their personal science journey & \\\hline

\end{tabular}
\end{center}


\end{document}
